\hypertarget{a-concise-introduction-to-generics-sop}{%
\section{\texorpdfstring{A concise introduction to
\texttt{generics-sop}}{A concise introduction to generics-sop}}\label{a-concise-introduction-to-generics-sop}}

A fascinating aspect of Lisp-based programming languages is that
\href{https://en.wikipedia.org/wiki/Code_as_data}{code is data and data
is code}. This property, called
\href{https://en.wikipedia.org/wiki/Homoiconicity}{homoiconicity}, is
what makes Lisp macros so powerful. This sort of runtime operation done
on arbitrary datatypes is called
\href{https://en.wikipedia.org/wiki/Polymorphism_(computer_science)\#Polytypism}{polytypism}
or datatype genericity. In Haskell, the following two packages provide
datatype genericity:

\begin{enumerate}
\def\labelenumi{\arabic{enumi}.}
\tightlist
\item
  \href{https://hackage.haskell.org/package/base-4.16.0.0/docs/GHC-Generics.html}{GHC
  Generics}
\item
  \href{https://hackage.haskell.org/package/generics-sop}{generics-sop}
\end{enumerate}

While the former comes with \texttt{base}, the latter is much easier to
write generics code in, and this blog post introduces
\texttt{generics-sop}.

\begin{itemize}
\tightlist
\item
  \protect\hyperlink{a-concise-introduction-to-generics-sop}{A concise
  introduction to \texttt{generics-sop}}

  \begin{itemize}
  \tightlist
  \item
    \protect\hyperlink{motivation}{Motivation}
  \item
    \protect\hyperlink{basics}{Basics}

    \begin{itemize}
    \tightlist
    \item
      \protect\hyperlink{datatypes-are-sops-under-the-hood}{Datatypes
      are SOPs under the hood}
    \item
      \protect\hyperlink{sops-are-tables}{SOPs are tables}
    \item
      \protect\hyperlink{interlude-a-foray-into-type-level-programming}{Interlude:
      a foray into type-level programming}
    \end{itemize}
  \item
    \protect\hyperlink{lets-play-with-sops}{Let's play with SOPs}

    \begin{itemize}
    \tightlist
    \item
      \protect\hyperlink{interlude-ns--np}{Interlude: \texttt{NS} \&
      \texttt{NP}}
    \item
      \protect\hyperlink{code-as-data-data-as-code}{Code as data; data
      as code}
    \end{itemize}
  \item
    \protect\hyperlink{example-1-generic-equality}{Example 1: generic
    equality}

    \begin{itemize}
    \tightlist
    \item
      \protect\hyperlink{naive-implementation}{Naive implementation}
    \item
      \protect\hyperlink{combinator-based-implementation}{Combinator-based
      implementation}

      \begin{itemize}
      \tightlist
      \item
        \protect\hyperlink{interlude-specialized-combinators}{Interlude:
        Specialized combinators}
      \end{itemize}
    \end{itemize}
  \item
    \protect\hyperlink{example-2-route-encoding}{Example 2: route
    encoding}

    \begin{itemize}
    \tightlist
    \item
      \protect\hyperlink{manual-implementation}{Manual implementation}
    \item
      \protect\hyperlink{identify-the-general-pattern}{Identify the
      general pattern}
    \item
      \protect\hyperlink{write-the-generic-version}{Write the generic
      version}
    \end{itemize}
  \item
    \protect\hyperlink{example-3-route-decoding}{Example 3: route
    decoding}

    \begin{itemize}
    \tightlist
    \item
      \protect\hyperlink{slist}{\texttt{SList}}
    \item
      \protect\hyperlink{anamomrphism-combinators}{Anamomrphism
      combinators}
    \item
      \protect\hyperlink{implement-gdecoderoute}{Implement
      \texttt{gDecodeRoute}}
    \item
      \protect\hyperlink{putting-it-all-together}{Putting it all
      together}
    \end{itemize}
  \item
    \protect\hyperlink{further-information}{Further information}
  \end{itemize}
\end{itemize}

\hypertarget{motivation}{%
\subsection{Motivation}\label{motivation}}

Generic programming is useful in avoiding writing of boilerplate
implementations for each datatype that are otherwise similar in some
way. This could be a
(\href{https://www.sciencedirect.com/science/article/pii/S0167642304000152}{polytypic})
function or a typeclass instance. For example, instead of having to
manually write \texttt{FromJSON} and \texttt{ToJSON} instances for each
of your datatypes, you can use generics to derive them automatically.
Other examples include pretty printers, parsers, equality functions and
route encoders.

\hypertarget{basics}{%
\subsection{Basics}\label{basics}}

Before diving, we must understand the ``SOP'' in generics-\textbf{sop}.

\hypertarget{datatypes-are-sops-under-the-hood}{%
\subsubsection{Datatypes are SOPs under the
hood}\label{datatypes-are-sops-under-the-hood}}

Haskell has two kinds of datatypes:

\begin{enumerate}
\def\labelenumi{\arabic{enumi}.}
\tightlist
\item
  \href{https://en.wikipedia.org/wiki/Algebraic_data_type}{Algebraic
  data types}, or ADTs
\item
  Newtypes
\end{enumerate}

Both ADTs and newtypes are a ``sum-of-product'' (SOP) under the hood.
When writing generics-sop code, we operate on these SOPs rather than
directly on the datatype, because every datatype is ``polymorphic'' in
their SOP representation. The basic idea is that if you can write a
function \texttt{SOP\ -\textgreater{}\ a}, then you get
\texttt{SomeDataType\ -\textgreater{}\ a} for free for \emph{any}
\texttt{SomeDataType}. This is called
\href{https://en.wikipedia.org/wiki/Polymorphism_(computer_science)\#Polytypism}{polytypism}.

Consider the following ADT (from
\href{https://hackage.haskell.org/package/these-1.1.1.1/docs/Data-These.html}{the
\texttt{these} package}):

\begin{Shaded}
\begin{Highlighting}[]
\CommentTok{{-}{-} \textasciigrave{}These\textasciigrave{} is like \textasciigrave{}Either\textasciigrave{}, but with a 3rd possibility of representing both values.}
\KeywordTok{data} \DataTypeTok{These}\NormalTok{ a b}
  \OtherTok{=} \DataTypeTok{This}\NormalTok{ a }
  \OperatorTok{|} \DataTypeTok{That}\NormalTok{ b}
  \OperatorTok{|} \DataTypeTok{These}\NormalTok{ a b}
\end{Highlighting}
\end{Shaded}

Here, \texttt{These} is a
\href{https://en.wikipedia.org/wiki/Tagged_union}{\textbf{sum} type},
with \texttt{This}, \texttt{That} and \texttt{These} being its three sum
constructors. Each sum constructor themselves are
\href{https://en.wikipedia.org/wiki/Product_type}{\textbf{product}
types} - inasmuch as, say, the \texttt{a} and \texttt{b} in the 3rd
constructor together represent a product type associated with
\emph{that} constructor. The type \texttt{These} is therefore a ``sum of
product''.

\hypertarget{sops-are-tables}{%
\subsubsection{SOPs are tables}\label{sops-are-tables}}

To gain better intuition, we may visualize the \texttt{These} SOP in a
table form:

\begin{longtable}[]{@{}llll@{}}
\toprule
Constructor & Arg 1 & Arg 2 & \ldots{} \\
\midrule
\endhead
This & a & & \\
That & b & & \\
These & a & b & \\
\bottomrule
\end{longtable}

As every Haskell datatype is a SOP, they can be (visually) reduced to a
table like the above. Each row represents the sum constructor, and the
individual cells to the right represent the arguments to the
constructors (product type). We can drop the constructor \emph{names}
entirely and simplify the table as:

\begin{longtable}[]{@{}ll@{}}
\toprule
\endhead
a & \\
b & \\
a & b \\
\bottomrule
\end{longtable}

(\emph{\texttt{These}} type visually represented as a table)

Every cell in this table is a unique type. To define this table in
Haskell we could use type-level lists, specifically a type-level
\emph{list of lists}. The outer list represents the sum constructor, and
the inner list represents the products. The \emph{kind} of this table
type would then be \texttt{{[}{[}Type{]}{]}}. Indeed
\href{https://hackage.haskell.org/package/generics-sop-0.5.1.2/docs/Generics-SOP.html\#t:Code}{this}
is what generics-sop uses. We can define the table type for
\texttt{These} in Haskell as follows:

\begin{Shaded}
\begin{Highlighting}[]
\KeywordTok{type} \DataTypeTok{TheseTable}\NormalTok{ a b }\OtherTok{=}
\NormalTok{  \textquotesingle{}[ \textquotesingle{}[ a ]}
\NormalTok{     \textquotesingle{}[ b ]}
\NormalTok{     \textquotesingle{}[ a, b ]}
\NormalTok{  ]}
\end{Highlighting}
\end{Shaded}

If you are confused about this syntax, read the following ``Interlude''
section.

\hypertarget{interlude-a-foray-into-type-level-programming}{%
\subsubsection{Interlude: a foray into type-level
programming}\label{interlude-a-foray-into-type-level-programming}}

What is a ``kind''? Kinds are to types what types are to terms. For
example, the \emph{type of} of the term \texttt{"Hello\ world"} is
\texttt{String}. The latter is a ``type,'' whereas the former is a
``term''. Furthermore we can go one level up and ask - what the
\emph{kind of} of the type \texttt{String} is; and the answer is
\texttt{Type}. We can clarify this further by explicitly annotating the
kinds of types when defining them (just as we annotate the types of
terms when defining them):

\begin{Shaded}
\begin{Highlighting}[]
\CommentTok{{-}{-} Here, we define a term (2nd line) and declare its type (1st line)}
\OtherTok{someBool ::} \DataTypeTok{Bool}
\NormalTok{someBool }\OtherTok{=} \DataTypeTok{True} 

\CommentTok{{-}{-} Here, we define a type (2nd line) and declare its kind (1st line)}
\KeywordTok{type} \DataTypeTok{Bool}\OtherTok{ ::} \DataTypeTok{Type}
\KeywordTok{data} \DataTypeTok{Bool} \OtherTok{=} \DataTypeTok{False} \OperatorTok{|} \DataTypeTok{True}
\end{Highlighting}
\end{Shaded}

Parametrized types, such as \texttt{Maybe}, have a type-level function
as their kind:

\begin{Shaded}
\begin{Highlighting}[]
\KeywordTok{type} \DataTypeTok{Maybe}\OtherTok{ ::} \DataTypeTok{Type} \OtherTok{{-}\textgreater{}} \DataTypeTok{Type} 
\KeywordTok{data} \DataTypeTok{Maybe}\NormalTok{ a }\OtherTok{=} \DataTypeTok{Nothing} \OperatorTok{|} \DataTypeTok{Just}\NormalTok{ a}
\end{Highlighting}
\end{Shaded}

Here we say that ``the type \texttt{Maybe} is of kind
\texttt{Type\ -\textgreater{}\ Type}''. In other words, \texttt{Maybe}
is a \emph{type-level function} that takes a type of kind \texttt{Type}
as an argument and returns another type of the same kind \texttt{Type}
as its result.

Finally, we are now in a position to understand the kind of
\texttt{TheseTable} described in the prior section:

\begin{Shaded}
\begin{Highlighting}[]
\KeywordTok{type} \DataTypeTok{TheseTable}\OtherTok{ ::} \DataTypeTok{Type} \OtherTok{{-}\textgreater{}} \DataTypeTok{Type} \OtherTok{{-}\textgreater{}}\NormalTok{ [[}\DataTypeTok{Type}\NormalTok{]]}
\KeywordTok{type} \DataTypeTok{TheseTable}\NormalTok{ a b }\OtherTok{=}
\NormalTok{  \textquotesingle{}[ \textquotesingle{}[ a ]}
\NormalTok{     \textquotesingle{}[ b ]}
\NormalTok{     \textquotesingle{}[ a, b ]}
\NormalTok{  ]}
\end{Highlighting}
\end{Shaded}

\texttt{{[}Type{]}} is the kind of type-level lists, and
\texttt{{[}{[}Type{]}{]}} is the kind of type-level lists of lists. The
tick (\texttt{\textquotesingle{}}) lifts a term into a type. So, while
\texttt{True} represents a \emph{term} of type \texttt{Bool},
\texttt{\textquotesingle{}True} on the other hand represents a
\emph{type} of \emph{kind} \texttt{Bool}, just as
\texttt{\textquotesingle{}{[}a{]}} represents a type of the kind
\texttt{{[}Type{]}}. The tick ``promotes'' a term to be a type. See
\href{https://ghc.gitlab.haskell.org/ghc/doc/users_guide/exts/data_kinds.html\#}{Datatype
promotion} in GHC user guide for details.

See
\href{https://lexi-lambda.github.io/blog/2021/03/25/an-introduction-to-typeclass-metaprogramming/}{An
introduction to typeclass metaprogramming} as well as
\href{https://thinkingwithtypes.com/}{Thinking with Types} for more on
type-level programming.

\hypertarget{lets-play-with-sops}{%
\subsection{Let's play with SOPs}\label{lets-play-with-sops}}

Enough theory, let's get our hands dirty in GHCi. If you use Nix, you
can clone \href{https://github.com/srid/generics-sop-examples}{this
repo} and run \texttt{bin/repl} to get GHCi with everything configured
ahead for you.

\begin{verbatim}
$ git clone https://github.com/srid/generics-sop-examples.git 
$ cd ./generics-sop-examples
$ bin/repl
[1 of 1] Compiling Main             ( src/Main.hs, interpreted )
Ok, one module loaded.
*Main>
\end{verbatim}

The project already has \texttt{generics-sop} and \texttt{sop-core}
added to the .cabal file, so you should be able to import it:

\begin{Shaded}
\begin{Highlighting}[]
\OperatorTok{\textgreater{}} \KeywordTok{import} \DataTypeTok{Generics.SOP}
\end{Highlighting}
\end{Shaded}

We also have
\href{https://hackage.haskell.org/package/these-1.1.1.1/docs/Data-These.html}{the
\texttt{these} package} added to the .cabal file because it provides the
above \texttt{These} type from \texttt{Data.These} module. To explore
the SOP representation of the \texttt{These} type, let us do some
bootstrapping:

\begin{Shaded}
\begin{Highlighting}[]
\OperatorTok{\textgreater{}} \KeywordTok{import} \DataTypeTok{Data.These} 
\OperatorTok{\textgreater{}} \KeywordTok{instance} \DataTypeTok{Generic}\NormalTok{ (}\DataTypeTok{These}\NormalTok{ a b) }\CommentTok{{-}{-} Derive generics{-}sop instance}
\OperatorTok{\textgreater{}} \KeywordTok{let}\NormalTok{ breakfast }\OtherTok{=} \DataTypeTok{These} \StringTok{"Egg"} \StringTok{"Sausage"}\OtherTok{ ::} \DataTypeTok{These} \DataTypeTok{String} \DataTypeTok{String}
\end{Highlighting}
\end{Shaded}

We derived \texttt{Generic} on the type and created a term value called
\texttt{breakfast} (we are eating both eggs and sausages). To get the
SOP representation of this value, we can use
\href{https://hackage.haskell.org/package/generics-sop-0.5.1.2/docs/Generics-SOP.html\#v:from}{\texttt{from}}:

\begin{Shaded}
\begin{Highlighting}[]
\OperatorTok{\textgreater{}}\NormalTok{ unSOP }\OperatorTok{.}\NormalTok{ from }\OperatorTok{$}\NormalTok{ breakfast}
\DataTypeTok{S}\NormalTok{ (}\DataTypeTok{S}\NormalTok{ (}\DataTypeTok{Z}\NormalTok{ (}\DataTypeTok{I} \StringTok{"Egg"} \OperatorTok{:*} \DataTypeTok{I} \StringTok{"Sausage"} \OperatorTok{:*} \DataTypeTok{Nil}\NormalTok{)))}
\end{Highlighting}
\end{Shaded}

Well explain the above value structure in a bit, but the key thing to
realize is that this value corresponds to the 3rd row in the SOP table
for \texttt{These}:

\begin{longtable}[]{@{}ll@{}}
\toprule
\endhead
String & \\
String & \\
String & String \\
\bottomrule
\end{longtable}

Because \texttt{breakfast} is a value of the 3rd constructor of
\texttt{These} and it contains two values (the product of ``Egg'' and
``Sausage''). The corresponding Haskell type for this table:

\begin{Shaded}
\begin{Highlighting}[]
\KeywordTok{type} \DataTypeTok{TheseTable}\OtherTok{ ::}\NormalTok{ [[}\DataTypeTok{Type}\NormalTok{]]}
\KeywordTok{type} \DataTypeTok{TheseTable} \OtherTok{=}
\NormalTok{  \textquotesingle{}[ \textquotesingle{}[ }\DataTypeTok{String}\NormalTok{ ]}
\NormalTok{     \textquotesingle{}[ }\DataTypeTok{String}\NormalTok{ ]}
\NormalTok{     \textquotesingle{}[ }\DataTypeTok{String}\NormalTok{, }\DataTypeTok{String}\NormalTok{ ]}
\NormalTok{  ]}
\end{Highlighting}
\end{Shaded}

This type is automatically provided by \texttt{generics-sop} whenever we
derive a \texttt{Generic} instance for the type in question. We did
precisely that further above by evaluating
\texttt{instance\ Generic\ (These\ a\ b)} in GHCi. Instead of manually
defining \texttt{TheseTable} as above, deriving \texttt{Generic} gives
it for free, in the form of \texttt{Code\ a}
(viz.~\texttt{Code\ (These\ a\ b)}).

\begin{Shaded}
\begin{Highlighting}[]
\OperatorTok{\textgreater{}} \OperatorTok{:}\NormalTok{k }\DataTypeTok{Code}\NormalTok{ (}\DataTypeTok{These} \DataTypeTok{String} \DataTypeTok{String}\NormalTok{)}
\DataTypeTok{Code}\NormalTok{ (}\DataTypeTok{These} \DataTypeTok{String} \DataTypeTok{String}\NormalTok{)}\OtherTok{ ::}\NormalTok{ [[}\DataTypeTok{Type}\NormalTok{]]}
\end{Highlighting}
\end{Shaded}

In brief, remember this: \texttt{Code\ a} gives us the SOP table
\emph{type} for the datatype \texttt{a}. Now how do we get the SOP table
\emph{value}? That is what \texttt{from} is for:

\begin{Shaded}
\begin{Highlighting}[]
\OperatorTok{\textgreater{}} \OperatorTok{:}\NormalTok{t (unSOP }\OperatorTok{.}\NormalTok{ from }\OperatorTok{$}\NormalTok{ breakfast)}
\NormalTok{(unSOP }\OperatorTok{.}\NormalTok{ from }\OperatorTok{$}\NormalTok{ breakfast)}
\OtherTok{  ::} \DataTypeTok{NS}
       \OperatorTok{@}\NormalTok{[}\DataTypeTok{Type}\NormalTok{]}
\NormalTok{       (}\DataTypeTok{NP} \OperatorTok{@}\DataTypeTok{Type} \DataTypeTok{I}\NormalTok{)}
\NormalTok{       ((\textquotesingle{}}\OperatorTok{:}\NormalTok{)}
          \OperatorTok{@}\NormalTok{[}\DataTypeTok{Type}\NormalTok{]}
\NormalTok{          ((\textquotesingle{}}\OperatorTok{:}\NormalTok{) }\OperatorTok{@}\DataTypeTok{Type}\NormalTok{ [}\DataTypeTok{Char}\NormalTok{] (\textquotesingle{}[] }\OperatorTok{@}\DataTypeTok{Type}\NormalTok{))}
\NormalTok{          ((\textquotesingle{}}\OperatorTok{:}\NormalTok{)}
             \OperatorTok{@}\NormalTok{[}\DataTypeTok{Type}\NormalTok{]}
\NormalTok{             ((\textquotesingle{}}\OperatorTok{:}\NormalTok{) }\OperatorTok{@}\DataTypeTok{Type}\NormalTok{ [}\DataTypeTok{Char}\NormalTok{] (\textquotesingle{}[] }\OperatorTok{@}\DataTypeTok{Type}\NormalTok{))}
\NormalTok{             ((\textquotesingle{}}\OperatorTok{:}\NormalTok{)}
                \OperatorTok{@}\NormalTok{[}\DataTypeTok{Type}\NormalTok{]}
\NormalTok{                ((\textquotesingle{}}\OperatorTok{:}\NormalTok{) }\OperatorTok{@}\DataTypeTok{Type}\NormalTok{ [}\DataTypeTok{Char}\NormalTok{] ((\textquotesingle{}}\OperatorTok{:}\NormalTok{) }\OperatorTok{@}\DataTypeTok{Type}\NormalTok{ [}\DataTypeTok{Char}\NormalTok{] (\textquotesingle{}[] }\OperatorTok{@}\DataTypeTok{Type}\NormalTok{)))}
\NormalTok{                (\textquotesingle{}[] }\OperatorTok{@}\NormalTok{[}\DataTypeTok{Type}\NormalTok{]))))}
\end{Highlighting}
\end{Shaded}

That is quite a mouthful because type-level lists are not represented
cleanly in GHCi. But we can reduce it (in our mind) to the following

\begin{Shaded}
\begin{Highlighting}[]
\OperatorTok{\textgreater{}} \OperatorTok{:}\NormalTok{t (unSOP }\OperatorTok{.}\NormalTok{ from }\OperatorTok{$}\NormalTok{ breakfast)}
\NormalTok{(unSOP }\OperatorTok{.}\NormalTok{ from }\OperatorTok{$}\NormalTok{ breakfast)}
\OtherTok{  ::} \DataTypeTok{NS}\NormalTok{ (}\DataTypeTok{NP} \DataTypeTok{I}\NormalTok{) \textquotesingle{}[ [}\DataTypeTok{String}\NormalTok{], [}\DataTypeTok{String}\NormalTok{], [}\DataTypeTok{String}\NormalTok{, }\DataTypeTok{String}\NormalTok{] ]}
\end{Highlighting}
\end{Shaded}

Notice how this is more or less isomorphic to our \texttt{TheseTable}
definition above. We will explain the \texttt{NS} and \texttt{NP} parts
next.

\hypertarget{interlude-ns-np}{%
\subsubsection{\texorpdfstring{Interlude: \texttt{NS} \&
\texttt{NP}}{Interlude: NS \& NP}}\label{interlude-ns-np}}

You wonder what the \texttt{NS\ (NP\ I)} part refers to in our table
type above. \texttt{NS} is a n-ary sum; and \texttt{NP} an n-ary
product. These are explained well in section 2 of
\href{https://github.com/kosmikus/SSGEP/blob/master/LectureNotes.pdf}{Applying
Type-Level and Generic Programming in Haskell}, but for our purposes -
you can treat \texttt{NS} as similar to the \texttt{Nat} type from the
\href{https://hackage.haskell.org/package/fin-0.2.1/docs/Data-Nat.html}{fin
package}, and \texttt{NP} as being similar to the \texttt{Vec} type from
the
\href{https://hackage.haskell.org/package/vec-0.4.1/docs/Data-Vec-Lazy.html\#t:Vec}{vec
package}.

The difference is that unlike \texttt{Vec} (a homogenous list),
\texttt{NP} is a heterogenous list whose element types are specified by
a type-level list.

\begin{Shaded}
\begin{Highlighting}[]
\OperatorTok{\textgreater{}} \OperatorTok{:}\NormalTok{k }\DataTypeTok{NP} \DataTypeTok{I}\NormalTok{ \textquotesingle{}[}\DataTypeTok{String}\NormalTok{, }\DataTypeTok{Int}\NormalTok{]}
\DataTypeTok{NP} \DataTypeTok{I}\NormalTok{ \textquotesingle{}[}\DataTypeTok{String}\NormalTok{, }\DataTypeTok{Int}\NormalTok{]}\OtherTok{ ::} \DataTypeTok{Type}
\end{Highlighting}
\end{Shaded}

Just like \texttt{Vec} can enforce the size of its homogenous list,
\texttt{NP} is a heterogenous list of exactly size 2. However, unlike
\texttt{Vec} we also say that the first element is of type
\texttt{String} and the second (and the last) element is of type
\texttt{Int} (hence a heterogenous list). To create a value of this
heterogeneous list:

\begin{Shaded}
\begin{Highlighting}[]
\OperatorTok{\textgreater{}} \DataTypeTok{I} \StringTok{"Meaning"} \OperatorTok{:*} \DataTypeTok{I} \DecValTok{42} \OperatorTok{:*} \DataTypeTok{Nil}\OtherTok{  ::} \DataTypeTok{NP} \DataTypeTok{I}\NormalTok{ \textquotesingle{}[}\DataTypeTok{String}\NormalTok{, }\DataTypeTok{Int}\NormalTok{]}
\DataTypeTok{I} \StringTok{"Meaning"} \OperatorTok{:*} \DataTypeTok{I} \DecValTok{42} \OperatorTok{:*} \DataTypeTok{Nil}
\end{Highlighting}
\end{Shaded}

This syntax should be unsurprising because \texttt{Nil} and
\texttt{(:*)} are the constructors of the \texttt{NP} type:

\begin{Shaded}
\begin{Highlighting}[]
\OperatorTok{\textgreater{}} \OperatorTok{:}\NormalTok{info }\DataTypeTok{NP}
\KeywordTok{data} \DataTypeTok{NP}\OtherTok{ ::}\NormalTok{ (k }\OtherTok{{-}\textgreater{}} \DataTypeTok{Type}\NormalTok{) }\OtherTok{{-}\textgreater{}}\NormalTok{ [k] }\OtherTok{{-}\textgreater{}} \DataTypeTok{Type} \KeywordTok{where}
  \DataTypeTok{Nil}\OtherTok{  ::} \DataTypeTok{NP}\NormalTok{ f \textquotesingle{}[]}
\OtherTok{  (:*) ::}\NormalTok{ f x }\OtherTok{{-}\textgreater{}} \DataTypeTok{NP}\NormalTok{ f xs }\OtherTok{{-}\textgreater{}} \DataTypeTok{NP}\NormalTok{ f (x \textquotesingle{}}\OperatorTok{:}\NormalTok{ xs)}
\end{Highlighting}
\end{Shaded}

(\href{https://hackage.haskell.org/package/sop-core-0.5.0.2/docs/Data-SOP.html\#t:NP}{View
haddocks})

The \texttt{I} is the identity functor, but it could also be something
else like \texttt{Maybe}:

\begin{Shaded}
\begin{Highlighting}[]
\OperatorTok{\textgreater{}} \DataTypeTok{Nothing} \OperatorTok{:*} \DataTypeTok{Just} \DecValTok{42} \OperatorTok{:*} \DataTypeTok{Nil}\OtherTok{  ::} \DataTypeTok{NP} \DataTypeTok{Maybe}\NormalTok{ \textquotesingle{}[}\DataTypeTok{String}\NormalTok{, }\DataTypeTok{Int}\NormalTok{]}
\DataTypeTok{Nothing} \OperatorTok{:*} \DataTypeTok{Just} \DecValTok{42} \OperatorTok{:*} \DataTypeTok{Nil}
\end{Highlighting}
\end{Shaded}

\texttt{NS} is the same, except now we are representing the same
characteristics (heterogeneity) but for the sum type instead of a
product type. A sum of length `n' over some functor `f':

\begin{Shaded}
\begin{Highlighting}[]
\OperatorTok{\textgreater{}} \OperatorTok{:}\NormalTok{info }\DataTypeTok{NS} 
\KeywordTok{data} \DataTypeTok{NS}\OtherTok{ ::}\NormalTok{ (k }\OtherTok{{-}\textgreater{}} \DataTypeTok{Type}\NormalTok{) }\OtherTok{{-}\textgreater{}}\NormalTok{ [k] }\OtherTok{{-}\textgreater{}} \DataTypeTok{Type} \KeywordTok{where}
  \DataTypeTok{Z}\OtherTok{ ::}\NormalTok{ f x }\OtherTok{{-}\textgreater{}} \DataTypeTok{NS}\NormalTok{ f (x \textquotesingle{}}\OperatorTok{:}\NormalTok{ xs)}
  \DataTypeTok{S}\OtherTok{ ::} \DataTypeTok{NS}\NormalTok{ f xs }\OtherTok{{-}\textgreater{}} \DataTypeTok{NS}\NormalTok{ f (x \textquotesingle{}}\OperatorTok{:}\NormalTok{ xs)}
\end{Highlighting}
\end{Shaded}

(\href{https://hackage.haskell.org/package/sop-core-0.5.0.2/docs/Data-SOP.html\#t:NS}{View
haddocks})

When the value is \texttt{Z} it indicates the first sum constructor;
when it is \texttt{S\ .\ Z} it is the second, and so on. Our
\texttt{breakfast} value above uses \texttt{These}, which is the 3rd
constructor. So, to construct the SOP representation of this value
directly, we would use \texttt{S\ .\ S\ .\ Z}. This is exactly what we
saw above (repeated here):

\begin{Shaded}
\begin{Highlighting}[]
\CommentTok{{-}{-} Note the \textasciigrave{}S . S . Z\textasciigrave{}}
\OperatorTok{\textgreater{}}\NormalTok{ unSOP }\OperatorTok{.}\NormalTok{ from }\OperatorTok{$}\NormalTok{ breakfast}
\DataTypeTok{S}\NormalTok{ (}\DataTypeTok{S}\NormalTok{ (}\DataTypeTok{Z}\NormalTok{ (}\DataTypeTok{I} \StringTok{"Egg"} \OperatorTok{:*} \DataTypeTok{I} \StringTok{"Sausage"} \OperatorTok{:*} \DataTypeTok{Nil}\NormalTok{)))}
\OperatorTok{\textgreater{}}
\OperatorTok{\textgreater{}} \OperatorTok{:}\NormalTok{t (unSOP }\OperatorTok{.}\NormalTok{ from }\OperatorTok{$}\NormalTok{ breakfast)}
\NormalTok{(unSOP }\OperatorTok{.}\NormalTok{ from }\OperatorTok{$}\NormalTok{ breakfast)}
\OtherTok{  ::} \DataTypeTok{NS}\NormalTok{ (}\DataTypeTok{NP} \DataTypeTok{I}\NormalTok{) \textquotesingle{}[ [}\DataTypeTok{String}\NormalTok{], [}\DataTypeTok{String}\NormalTok{], [}\DataTypeTok{String}\NormalTok{, }\DataTypeTok{String}\NormalTok{] ]}
\end{Highlighting}
\end{Shaded}

\texttt{NS}'s functor is a \texttt{NP\ I}, and so the inner value of
that sum choice is an n-ary product (remember: we are working with a
sum-of-product), whose value is
\texttt{I\ "Egg"\ :*\ I\ "Sausage"\ :*\ Nil}. Putting that product
inside a sum, we get
\texttt{S\ (S\ (Z\ (I\ "Egg"\ :*\ I\ "Sausage"\ :*\ Nil)))}.

\hypertarget{code-as-data-data-as-code}{%
\subsubsection{Code as data; data as
code}\label{code-as-data-data-as-code}}

Ther SOP representation of \texttt{These} can be manually constructed.
First we build the constructor arguments (product), and then we build
the constructor itself (sum):

\begin{Shaded}
\begin{Highlighting}[]
\OperatorTok{\textgreater{}} \KeywordTok{let}\NormalTok{ prod }\OtherTok{=} \DataTypeTok{I} \StringTok{"Egg"} \OperatorTok{:*} \DataTypeTok{I} \StringTok{"Sausage"} \OperatorTok{:*} \DataTypeTok{Nil}\OtherTok{ ::} \DataTypeTok{NP} \DataTypeTok{I}\NormalTok{ \textquotesingle{}[}\DataTypeTok{String}\NormalTok{, }\DataTypeTok{String}\NormalTok{]}
\OperatorTok{\textgreater{}} \KeywordTok{let} \FunctionTok{sum} \OtherTok{=} \DataTypeTok{S} \OperatorTok{$} \DataTypeTok{S} \OperatorTok{$} \DataTypeTok{Z}\OtherTok{ prod ::} \DataTypeTok{NP} \DataTypeTok{I}\NormalTok{ \textquotesingle{}[[}\DataTypeTok{String}\NormalTok{], [}\DataTypeTok{String}\NormalTok{], [}\DataTypeTok{String}\NormalTok{, }\DataTypeTok{String}\NormalTok{]]}
\OperatorTok{\textgreater{}} \OperatorTok{:}\NormalTok{t }\FunctionTok{sum} 
\FunctionTok{sum}\OtherTok{ ::} \DataTypeTok{NS}\NormalTok{ (}\DataTypeTok{NP} \DataTypeTok{I}\NormalTok{) \textquotesingle{}[[}\DataTypeTok{String}\NormalTok{], [}\DataTypeTok{String}\NormalTok{], [}\DataTypeTok{String}\NormalTok{, }\DataTypeTok{String}\NormalTok{]]}
\end{Highlighting}
\end{Shaded}

And from this representation, we can produce a value of type
\texttt{These} easily, using
\href{https://hackage.haskell.org/package/generics-sop-0.5.1.2/docs/Generics-SOP.html\#v:to}{\texttt{to}}:

\begin{Shaded}
\begin{Highlighting}[]
\OperatorTok{\textgreater{}}\NormalTok{ to }\OperatorTok{@}\NormalTok{(}\DataTypeTok{These} \DataTypeTok{String} \DataTypeTok{String}\NormalTok{) (}\DataTypeTok{SOP} \FunctionTok{sum}\NormalTok{)}
\DataTypeTok{These} \StringTok{"Egg"} \StringTok{"Sausage"}
\end{Highlighting}
\end{Shaded}

Let us stop for a moment and reflect on what we just did. By treating
the type-definition of \texttt{These} (``code'') as a generic
\texttt{SOP} table (``data'') -- i.e., code as data -- we are able to
generate a value (``code'') for that type (``data'') -- ie., data as
code -- but without using the constructors of that type. This is generic
programming in Haskell; you program \emph{generically}, without being
privy to the actual type used.

This concludes the section on playing with SOPs. Now let us do something
useful.

\hypertarget{example-1-generic-equality}{%
\subsection{Example 1: generic
equality}\label{example-1-generic-equality}}

GHC's
\href{https://ghc.gitlab.haskell.org/ghc/doc/users_guide/exts/deriving_strategies.html\#deriving-strategies}{stock
deriving} can be used to derive instances for builtin type classes, like
\texttt{Eq}, on user-defined datatypes. This works for builtin type
classes, but generics-sop (as well as GHC.Generics) comes in handy when
you want to derive generically for arbitrary typeclasses. For a moment,
let us assume that GHC had no support for stock deriving; how will we
derive our \texttt{Eq} instance?

We want a function \texttt{geq} that takes \emph{any} datatype
\texttt{a} (this makes the function
\href{https://www.sciencedirect.com/science/article/pii/S0167642304000152}{polytypic}),
and does equality check on its arguments. In effect, we want:

\begin{Shaded}
\begin{Highlighting}[]
\OtherTok{geq ::} \DataTypeTok{Generic}\NormalTok{ a }\OtherTok{=\textgreater{}}\NormalTok{ a }\OtherTok{{-}\textgreater{}}\NormalTok{ a }\OtherTok{{-}\textgreater{}} \DataTypeTok{Bool}
\end{Highlighting}
\end{Shaded}

This function can be further broken down to operate on SOP structures
directly, so as to ``forget'' the specific \texttt{a}:

\begin{Shaded}
\begin{Highlighting}[]
\OtherTok{geq ::} \KeywordTok{forall}\NormalTok{ a}\OperatorTok{.} \DataTypeTok{Generic}\NormalTok{ a }\OtherTok{=\textgreater{}}\NormalTok{ a }\OtherTok{{-}\textgreater{}}\NormalTok{ a }\OtherTok{{-}\textgreater{}} \DataTypeTok{Bool}
\NormalTok{geq x y }\OtherTok{=}\NormalTok{ geq\textquotesingle{} }\OperatorTok{@}\NormalTok{a (unSOP }\OperatorTok{$}\NormalTok{ from x) (unSOP }\OperatorTok{$}\NormalTok{ from y)}

\OtherTok{geq\textquotesingle{} ::} \DataTypeTok{NS}\NormalTok{ (}\DataTypeTok{NP} \DataTypeTok{I}\NormalTok{) (}\DataTypeTok{Code}\NormalTok{ a) }\OtherTok{{-}\textgreater{}} \DataTypeTok{NS}\NormalTok{ (}\DataTypeTok{NP} \DataTypeTok{I}\NormalTok{) (}\DataTypeTok{Code}\NormalTok{ a) }\OtherTok{{-}\textgreater{}} \DataTypeTok{Bool}
\NormalTok{geq\textquotesingle{} }\OtherTok{=} \FunctionTok{undefined}
\end{Highlighting}
\end{Shaded}

Our problem has now been reduced to operating on SOP tables, and our
task is to implement \texttt{geq\textquotesingle{}}.

At this point you are probably thinking we can just case-match on the
arguments, but remember that the n-ary sum type \texttt{NS} is a
\href{https://downloads.haskell.org/~ghc/latest/docs/html/users_guide/exts/gadt.html\#gadt}{GADT}
(its type index is
\href{https://en.wikipedia.org/wiki/Dependent_type}{dependent} on the
sum constructor). We have to instead case-match at the \emph{type-level}
as it were. This is what type-classes are for. The general pattern is
that when wanting a \texttt{foo} that case-match'es at type-level, we
write a type-class \texttt{Foo} and write instances for each case-match
pattern.

\hypertarget{naive-implementation}{%
\subsubsection{Naive implementation}\label{naive-implementation}}

For pedagogic reasons, we begin with a naive implementation of
\texttt{geq\textquotesingle{}} to illustrate the above. We need a
\texttt{sumEq} function that checks equality of first constructor and
then recurses for others; it will case-match on the outer list.
Likewise, for each sum constructor, we will need a \texttt{prodEq} that
checks equality of its products; and it does so, similarly, by checking
equality of the first product and then recurses for the rest; it will
case-match on the inner list.

\begin{Shaded}
\begin{Highlighting}[]
\OtherTok{geq\textquotesingle{} ::} \DataTypeTok{SumEq}\NormalTok{ (}\DataTypeTok{Code}\NormalTok{ a) }\OtherTok{=\textgreater{}} \DataTypeTok{NS}\NormalTok{ (}\DataTypeTok{NP} \DataTypeTok{I}\NormalTok{) (}\DataTypeTok{Code}\NormalTok{ a) }\OtherTok{{-}\textgreater{}} \DataTypeTok{NS}\NormalTok{ (}\DataTypeTok{NP} \DataTypeTok{I}\NormalTok{) (}\DataTypeTok{Code}\NormalTok{ a) }\OtherTok{{-}\textgreater{}} \DataTypeTok{Bool}
\NormalTok{geq\textquotesingle{} }\OtherTok{=}\NormalTok{ sumEq}

\CommentTok{{-}{-} \textasciigrave{}xss\textasciigrave{} is a type{-}level list of lists; \textasciigrave{}Code a\textasciigrave{}}
\KeywordTok{class} \DataTypeTok{SumEq}\NormalTok{ xss }\KeywordTok{where}
\OtherTok{  sumEq ::} \DataTypeTok{NS}\NormalTok{ (}\DataTypeTok{NP} \DataTypeTok{I}\NormalTok{) xss }\OtherTok{{-}\textgreater{}} \DataTypeTok{NS}\NormalTok{ (}\DataTypeTok{NP} \DataTypeTok{I}\NormalTok{) xss }\OtherTok{{-}\textgreater{}} \DataTypeTok{Bool}

\KeywordTok{instance} \DataTypeTok{SumEq}\NormalTok{ \textquotesingle{}[] }\KeywordTok{where}
\NormalTok{  sumEq }\OtherTok{=}\NormalTok{ \textbackslash{}}\KeywordTok{case}

\KeywordTok{instance}\NormalTok{ (}\DataTypeTok{ProdEq}\NormalTok{ xs, }\DataTypeTok{SumEq}\NormalTok{ xss) }\OtherTok{=\textgreater{}} \DataTypeTok{SumEq}\NormalTok{ (xs \textquotesingle{}}\OperatorTok{:}\NormalTok{ xss) }\KeywordTok{where}
  \CommentTok{{-}{-} Both values are the same constructor; so check equality on their products,}
  \CommentTok{{-}{-} using \textasciigrave{}prodEq\textasciigrave{}.}
\NormalTok{  sumEq (}\DataTypeTok{Z}\NormalTok{ x) (}\DataTypeTok{Z}\NormalTok{ y) }\OtherTok{=}\NormalTok{ prodEq x y}
  \CommentTok{{-}{-} Recurse on next sum constructor.}
\NormalTok{  sumEq (}\DataTypeTok{S}\NormalTok{ x) (}\DataTypeTok{S}\NormalTok{ y) }\OtherTok{=}\NormalTok{ sumEq x y}
  \CommentTok{{-}{-} Mismatching sum constructor; equality check failed.}
\NormalTok{  sumEq \_ \_ }\OtherTok{=} \DataTypeTok{False}

\KeywordTok{class} \DataTypeTok{ProdEq}\NormalTok{ xs }\KeywordTok{where}
\OtherTok{  prodEq ::} \DataTypeTok{NP} \DataTypeTok{I}\NormalTok{ xs }\OtherTok{{-}\textgreater{}} \DataTypeTok{NP} \DataTypeTok{I}\NormalTok{ xs }\OtherTok{{-}\textgreater{}} \DataTypeTok{Bool}

\KeywordTok{instance} \DataTypeTok{ProdEq}\NormalTok{ \textquotesingle{}[] }\KeywordTok{where}
\NormalTok{  prodEq }\DataTypeTok{Nil} \DataTypeTok{Nil} \OtherTok{=} \DataTypeTok{True}

\KeywordTok{instance}\NormalTok{ (}\DataTypeTok{Eq}\NormalTok{ x, }\DataTypeTok{ProdEq}\NormalTok{ xs) }\OtherTok{=\textgreater{}} \DataTypeTok{ProdEq}\NormalTok{ (x \textquotesingle{}}\OperatorTok{:}\NormalTok{ xs) }\KeywordTok{where}
  \CommentTok{{-}{-} First product argument should be equal; then we recurse for rest of arguments.}
\NormalTok{  prodEq (x }\OperatorTok{:*}\NormalTok{ xs) (y }\OperatorTok{:*}\NormalTok{ ys) }\OtherTok{=}\NormalTok{ x }\OperatorTok{==}\NormalTok{ y }\OperatorTok{\&\&}\NormalTok{ prodEq xs ys}
\end{Highlighting}
\end{Shaded}

Notice how in the first instance for \texttt{SumEq} we are ``pattern
matching'' as it were at the type-level and defining the implementation
for the scenario of zero sum constructors (not inhabitable). Then,
inductively, we define the next instance using recursion. When both
arguments are at \texttt{Z}, we match their products, using
\texttt{prodEq} which is defined similarly. Otherwise, we recurse into
the successor constructor (the \texttt{x} in \texttt{S\ x}). The story
for \texttt{ProdEq} is similar.

Finally, we can test that it works:

\begin{Shaded}
\begin{Highlighting}[]
\OperatorTok{\textgreater{}}\NormalTok{ geq (}\DataTypeTok{This} \DataTypeTok{True}\NormalTok{) (}\DataTypeTok{That} \DataTypeTok{False}\NormalTok{)}
\DataTypeTok{False}
\OperatorTok{\textgreater{}}\NormalTok{ geq (}\DataTypeTok{These} \DecValTok{42} \StringTok{"Hello"}\NormalTok{) (}\DataTypeTok{These} \DecValTok{42} \StringTok{"Hello"}\OtherTok{ ::} \DataTypeTok{These} \DataTypeTok{Int} \DataTypeTok{String}\NormalTok{)}
\DataTypeTok{True}
\end{Highlighting}
\end{Shaded}

We just implemented an equality function that works for \emph{any}
datatype (with \texttt{Generic} instance).

\hypertarget{combinator-based-implementation}{%
\subsubsection{Combinator-based
implementation}\label{combinator-based-implementation}}

Hopefully, the above naive implementation is illustratory of how one can
``transform'' SOP structures straightforwardly using typeclasses. N-ary
sums and products need to be processed at type-level, so it is not
uncommon to write new type-classes to dispatch on their constructors, as
shown above. Typically, however, you do not have to do that because
\texttt{generics-sop} provides combinators for common operations. Here,
we will rewrite the above implementation using these combinators.

The combinators are explained in depth in
\href{https://github.com/kosmikus/SSGEP/blob/master/LectureNotes.pdf}{ATLGP}.
We will introduce a few in this post. The particular combinators we need
for \texttt{geq} are:

\begin{longtable}[]{@{}lll@{}}
\toprule
Combinator & Description & Typeclass it replaces \\
\midrule
\endhead
\texttt{hcliftA2} & Lift elements of a NP or NS using given function &
\texttt{ProdEq} \\
\texttt{hcollapse} & Convert heterogenous structure into homogenous
value & \texttt{ProdEq} \\
\texttt{ccompare\_NS} & Compare two \texttt{NS} values &
\texttt{SumEq} \\
\bottomrule
\end{longtable}

To appreciate the value of these particular combinators, notice the 3rd
column indicating the type-class it intends to replace. Withtout further
ado, here is the new (compact) implementation:

\begin{Shaded}
\begin{Highlighting}[]
\OtherTok{geq ::} \KeywordTok{forall}\NormalTok{ a}\OperatorTok{.}\NormalTok{ (}\DataTypeTok{Generic}\NormalTok{ a, }\DataTypeTok{All2} \DataTypeTok{Eq}\NormalTok{ (}\DataTypeTok{Code}\NormalTok{ a)) }\OtherTok{=\textgreater{}}\NormalTok{ a }\OtherTok{{-}\textgreater{}}\NormalTok{ a }\OtherTok{{-}\textgreater{}} \DataTypeTok{Bool}
\NormalTok{geq x y }\OtherTok{=}\NormalTok{ geq\textquotesingle{} }\OperatorTok{@}\NormalTok{a (from x) (from y)}

\OtherTok{geq\textquotesingle{} ::} \DataTypeTok{All2} \DataTypeTok{Eq}\NormalTok{ (}\DataTypeTok{Code}\NormalTok{ a) }\OtherTok{=\textgreater{}} \DataTypeTok{SOP} \DataTypeTok{I}\NormalTok{ (}\DataTypeTok{Code}\NormalTok{ a) }\OtherTok{{-}\textgreater{}} \DataTypeTok{SOP} \DataTypeTok{I}\NormalTok{ (}\DataTypeTok{Code}\NormalTok{ a) }\OtherTok{{-}\textgreater{}} \DataTypeTok{Bool}
\NormalTok{geq\textquotesingle{} (}\DataTypeTok{SOP}\NormalTok{ c1) (}\DataTypeTok{SOP}\NormalTok{ c2) }\OtherTok{=}
\NormalTok{  ccompare\_NS (}\DataTypeTok{Proxy} \OperatorTok{@}\NormalTok{(}\DataTypeTok{All} \DataTypeTok{Eq}\NormalTok{)) }\DataTypeTok{False}\NormalTok{ eqProd }\DataTypeTok{False}\NormalTok{ c1 c2}
  \KeywordTok{where}
\OtherTok{    eqProd ::} \DataTypeTok{All} \DataTypeTok{Eq}\NormalTok{ xs }\OtherTok{=\textgreater{}} \DataTypeTok{NP} \DataTypeTok{I}\NormalTok{ xs }\OtherTok{{-}\textgreater{}} \DataTypeTok{NP} \DataTypeTok{I}\NormalTok{ xs }\OtherTok{{-}\textgreater{}} \DataTypeTok{Bool}
\NormalTok{    eqProd p1 p2 }\OtherTok{=}
\NormalTok{      foldl\textquotesingle{} (}\OperatorTok{\&\&}\NormalTok{) }\DataTypeTok{True} \OperatorTok{$}
\NormalTok{        hcollapse }\OperatorTok{$}\NormalTok{ hcliftA2 (}\DataTypeTok{Proxy}\OtherTok{ ::} \DataTypeTok{Proxy} \DataTypeTok{Eq}\NormalTok{) eqTerm p1 p2}
      \KeywordTok{where}
\OtherTok{        eqTerm ::} \KeywordTok{forall}\NormalTok{ a}\OperatorTok{.} \DataTypeTok{Eq}\NormalTok{ a }\OtherTok{=\textgreater{}} \DataTypeTok{I}\NormalTok{ a }\OtherTok{{-}\textgreater{}} \DataTypeTok{I}\NormalTok{ a }\OtherTok{{-}\textgreater{}} \DataTypeTok{K} \DataTypeTok{Bool}\NormalTok{ a}
\NormalTok{        eqTerm a b }\OtherTok{=}
          \DataTypeTok{K} \OperatorTok{$}\NormalTok{ a }\OperatorTok{==}\NormalTok{ b}
\end{Highlighting}
\end{Shaded}

This code introduces two more aspects to \texttt{generics-sop}:

\begin{itemize}
\tightlist
\item
  \textbf{Constraint propagation}: When generically transforming SOP
  structures we want to be able to ``propagate'' inner constraints
  outwardly, and this is what the \texttt{Proxy} class is being used for
  here. \texttt{All\ c\ xs} simply is an alias for
  \texttt{(c\ x1,\ c\ x2,\ ...)} where \texttt{xs} is a type-level list;
  likewise, \texttt{All2\ c\ xss} is \texttt{c\ x11,\ c\ x12,\ ...}
  where \texttt{xss} is type-level list of lists (ie.,
  \texttt{Code\ a\ \textasciitilde{}\ xss}). Clearly, we want the
  \texttt{Eq} constraint in table elements to apply to the whole table
  row and thereon to the table itself. And \texttt{All2\ Eq\ (Code\ a)}
  on \texttt{geq\textquotesingle{}} specifies this.
\item
  \textbf{Constant functor}: The constant functor \texttt{K} is defined
  as \texttt{data\ K\ a\ b\ =\ K\ a}; it ``discards'' the second type
  parameter, always containing the first. Where you see
  \texttt{K\ Bool\ a} we are discarding the polymorphic \texttt{a} (the
  type of the cell in the table), and returning the (constant) type
  \texttt{Bool}. When we transform the structure to be over \texttt{K}
  (using \texttt{hcliftA2}), we are essentially making the structure
  \emph{homogenous} in its elements, which in turn allows us to
  ``collapse'' it using \texttt{hcollapse} to get a single value out of
  it (which is what we need to be the result of \texttt{geq}).
\end{itemize}

This is just a brief taste of generics-sop combinators. Read
\href{https://github.com/kosmikus/SSGEP/blob/master/LectureNotes.pdf}{ATLGP}
for details, and we shall introduce more in the examples below.

\hypertarget{interlude-specialized-combinators}{%
\paragraph{Interlude: Specialized
combinators}\label{interlude-specialized-combinators}}

Most combinators are polymorphic over the containing structure, and as
such their type signatures can be pretty complex to understand. For this
reason, you might want to begin with using their \emph{monomorphized}
versions which have simpler type signatures. For example, the
polymorphic combinator \texttt{hcollapse} has the following signature
that makes it possible to work with any structure (\texttt{NS} or a
\texttt{NP}, etc).

\begin{Shaded}
\begin{Highlighting}[]
\OtherTok{hcollapse ::} \DataTypeTok{SListIN}\NormalTok{ h xs }\OtherTok{=\textgreater{}}\NormalTok{ h (}\DataTypeTok{K}\NormalTok{ a) xs }\OtherTok{{-}\textgreater{}} \DataTypeTok{CollapseTo}\NormalTok{ h a}
\end{Highlighting}
\end{Shaded}

This signature is not particularly easier to understand if you are not
very familiar with the library. But the monomorphized versions, such as
that for \texttt{NS}, are more straightforward to understand:

\begin{Shaded}
\begin{Highlighting}[]
\OtherTok{collapse\_NP ::} \DataTypeTok{NP}\NormalTok{ (}\DataTypeTok{K}\NormalTok{ a) xs }\OtherTok{{-}\textgreater{}}\NormalTok{ [a]}
\end{Highlighting}
\end{Shaded}

These specialized versions typically are suffixed as above (i.e.,
\texttt{\_NP}).

\hypertarget{example-2-route-encoding}{%
\subsection{Example 2: route encoding}\label{example-2-route-encoding}}

In the first example above we demonstrated how to use generics-sop to
generically implement \texttt{eq}. Here, we will show a more interesting
example. Specifically, how to represent routes for a statically
generated site using algebraic data types. We will derive encoders
(\texttt{route\ -\textgreater{}\ FilePath}) for them automatically using
generics-sop.

Imagine you are writing a static site in Haskell\footnote{Using
  generators like \href{https://jaspervdj.be/hakyll/}{Hakyll} or
  \href{https://ema.srid.ca/}{Ema}} for your blog posts. Each ``route''
in that site corresponds to a generated \texttt{.html} file. We will use
ADTs to represent the routes:

\begin{Shaded}
\begin{Highlighting}[]
\KeywordTok{data} \DataTypeTok{Route}
  \OtherTok{=} \DataTypeTok{Route\_Index} \CommentTok{{-}{-} index.html}
  \OperatorTok{|} \DataTypeTok{Route\_Blog} \DataTypeTok{BlogRoute} \CommentTok{{-}{-} blog/*}

\KeywordTok{data} \DataTypeTok{BlogRoute}
  \OtherTok{=} \DataTypeTok{BlogRoute\_Index} \CommentTok{{-}{-} blog/index.html}
  \OperatorTok{|} \DataTypeTok{BlogRoute\_Post} \DataTypeTok{PostSlug} \CommentTok{{-}{-} blog/$\{slug\}.html}

\KeywordTok{newtype} \DataTypeTok{PostSlug} \OtherTok{=} \DataTypeTok{PostSlug}\NormalTok{ \{}\OtherTok{unPostSlug ::} \DataTypeTok{Text}\NormalTok{\}}
\end{Highlighting}
\end{Shaded}

To compute the path to the \texttt{.html} file for each route, we need a
function \texttt{encodeRoute\ ::\ r\ -\textgreater{}\ FilePath}. It is
worth creating a typeclass for it because we can recursively encode the
ADT:

\begin{Shaded}
\begin{Highlighting}[]
\CommentTok{{-}{-} Class of routes that can be encoded to a filename.}
\KeywordTok{class} \DataTypeTok{IsRoute}\NormalTok{ r }\KeywordTok{where}
\OtherTok{  encodeRoute ::}\NormalTok{ r }\OtherTok{{-}\textgreater{}} \DataTypeTok{FilePath}
\end{Highlighting}
\end{Shaded}

\hypertarget{manual-implementation}{%
\subsubsection{Manual implementation}\label{manual-implementation}}

Before writing generic implementation, it is always useful to write the
implementation ``by hand''. Doing so enables us to begin to build an
intuition for what the generic version will look like.

\begin{Shaded}
\begin{Highlighting}[]
\CommentTok{{-}{-} This instance will remain manual.}
\KeywordTok{instance} \DataTypeTok{IsRoute} \DataTypeTok{PostSlug} \KeywordTok{where}
\NormalTok{  encodeRoute (}\DataTypeTok{PostSlug}\NormalTok{ slug) }\OtherTok{=}\NormalTok{ T.unpack slug }\OperatorTok{\textless{}\textgreater{}} \StringTok{".html"}

\CommentTok{{-}{-} These instances eventually will be generalized.}
\KeywordTok{instance} \DataTypeTok{IsRoute} \DataTypeTok{BlogRoute} \KeywordTok{where}
\NormalTok{  encodeRoute }\OtherTok{=}\NormalTok{ \textbackslash{}}\KeywordTok{case}
    \DataTypeTok{BlogRoute\_Index} \OtherTok{{-}\textgreater{}} \StringTok{"index.html"}
    \DataTypeTok{BlogRoute\_Post}\NormalTok{ slug }\OtherTok{{-}\textgreater{}} \StringTok{"post"} \OperatorTok{\textless{}/\textgreater{}}\NormalTok{ encodeRoute slug}

\KeywordTok{instance} \DataTypeTok{IsRoute} \DataTypeTok{Route} \KeywordTok{where}
\NormalTok{  encodeRoute }\OtherTok{=}\NormalTok{ \textbackslash{}}\KeywordTok{case}
    \DataTypeTok{Route\_Index} \OtherTok{{-}\textgreater{}} \StringTok{"index.html"}
    \DataTypeTok{Route\_Blog}\NormalTok{ br }\OtherTok{{-}\textgreater{}} \StringTok{"blog"} \OperatorTok{\textless{}/\textgreater{}}\NormalTok{ encodeRoute br}
\end{Highlighting}
\end{Shaded}

There is nothing we can do about \texttt{PostSlug} instance because it
is not an ADT, but we \emph{do} want to generically implement
\texttt{encodeRoute} for both \texttt{BlogRoute} and \texttt{Route}
generically.

\hypertarget{identify-the-general-pattern}{%
\subsubsection{Identify the general
pattern}\label{identify-the-general-pattern}}

After writing the implementation manually, the next step is to make it
as general as possible. Try to extract the ``general pattern'' behind
these manual implementations. Looking at the instances above, we can
conclude the general pattern as follows:

\begin{itemize}
\tightlist
\item
  To encode \texttt{Foo\_Bar} in a datatype \texttt{Foo}, we drop the
  \texttt{Foo\_}, and take the \texttt{Bar}. Then we convert it to
  \texttt{bar.html}.
\item
  If a sum constructor has arguments, we check that it has exactly one
  argument (arity \textless=1). Then call \texttt{encodeRoute} on that
  argument, and append it to the constructor's encoding using
  \texttt{/}.

  \begin{itemize}
  \tightlist
  \item
    For example, to encode \texttt{BlogPost\_Post\ (PostSlug\ "hello")}
    we first encode the constructor as \texttt{"post"}, then encode the
    only argument as \texttt{encodeRoute\ (PostSlug\ "hello")} which
    reduces to \texttt{"hello.html"}, thus producing the encoding
    \texttt{"post/hello.html"}. Finally when encoding
    \texttt{Route\_Blog\ br} - this gets encoded ito
    \texttt{"blog/post/hello.html"}, inductively.
  \end{itemize}
\end{itemize}

\hypertarget{write-the-generic-version}{%
\subsubsection{Write the generic
version}\label{write-the-generic-version}}

Having identified the general pattern, we are now able to write the
generic version of \texttt{encodeRoute}. Keep in mind the above pattern
while you follow the code below.

\begin{Shaded}
\begin{Highlighting}[]
\OtherTok{gEncodeRoute ::} \DataTypeTok{Generic}\NormalTok{ r }\OtherTok{=\textgreater{}}\NormalTok{ r }\OtherTok{{-}\textgreater{}} \DataTypeTok{FilePath}
\NormalTok{gEncodeRoute }\OtherTok{=} \FunctionTok{undefined}
\end{Highlighting}
\end{Shaded}

To derive route encoding from the constructor name, we need the datatype
metadata (provided by \texttt{HasDatatypeInfo}) from generics-sop.
\texttt{constructorInfo\ .\ datatypeInfo} gives us the constructor
information, from which we will determine the final route encoding using
the \texttt{hindex} combinator. Effectively, this enables us to produce
\texttt{"foo.html"} from a sum constructor like \texttt{Route\_Foo}.

\begin{Shaded}
\begin{Highlighting}[]
\OtherTok{gEncodeRoute ::} \KeywordTok{forall}\NormalTok{ r}\OperatorTok{.}\NormalTok{ (}\DataTypeTok{Generic}\NormalTok{ r, }\DataTypeTok{All2} \DataTypeTok{IsRoute}\NormalTok{ (}\DataTypeTok{Code}\NormalTok{ r), }\DataTypeTok{All} \DataTypeTok{IsRouteProd}\NormalTok{ (}\DataTypeTok{Code}\NormalTok{ r), }\DataTypeTok{HasDatatypeInfo}\NormalTok{ r) }\OtherTok{=\textgreater{}}\NormalTok{ r }\OtherTok{{-}\textgreater{}} \DataTypeTok{FilePath}
\NormalTok{gEncodeRoute x }\OtherTok{=}\NormalTok{ gEncodeRoute\textquotesingle{} }\OperatorTok{@}\NormalTok{r (from x)}

\OtherTok{gEncodeRoute\textquotesingle{} ::} \KeywordTok{forall}\NormalTok{ r}\OperatorTok{.}\NormalTok{ (}\DataTypeTok{All2} \DataTypeTok{IsRoute}\NormalTok{ (}\DataTypeTok{Code}\NormalTok{ r), }\DataTypeTok{All} \DataTypeTok{IsRouteProd}\NormalTok{ (}\DataTypeTok{Code}\NormalTok{ r), }\DataTypeTok{HasDatatypeInfo}\NormalTok{ r) }\OtherTok{=\textgreater{}} \DataTypeTok{SOP} \DataTypeTok{I}\NormalTok{ (}\DataTypeTok{Code}\NormalTok{ r) }\OtherTok{{-}\textgreater{}} \DataTypeTok{FilePath}
\NormalTok{gEncodeRoute\textquotesingle{} (}\DataTypeTok{SOP}\NormalTok{ x) }\OtherTok{=}
  \CommentTok{{-}{-} Determine the contructor name and then strip its prefix.}
  \KeywordTok{let}\OtherTok{ ctorNames ::}\NormalTok{ [}\DataTypeTok{ConstructorName}\NormalTok{] }\OtherTok{=}
\NormalTok{        hcollapse }\OperatorTok{$}\NormalTok{ hmap (}\DataTypeTok{K} \OperatorTok{.}\NormalTok{ constructorName) }\OperatorTok{$}\NormalTok{ datatypeCtors }\OperatorTok{@}\NormalTok{r}
\NormalTok{      ctorName }\OtherTok{=}\NormalTok{ ctorNames }\OperatorTok{!!}\NormalTok{ hindex x}
\NormalTok{      ctorSuffix }\OtherTok{=}\NormalTok{ ctorStripPrefix }\OperatorTok{@}\NormalTok{r ctorName}
  \CommentTok{{-}{-} Encode the product argument, if any, otherwise end the route string with ".html"}
   \KeywordTok{in} \KeywordTok{case}\NormalTok{ hcollapse }\OperatorTok{$}\NormalTok{ hcmap (}\DataTypeTok{Proxy} \OperatorTok{@}\DataTypeTok{IsRouteProd}\NormalTok{) encProd x }\KeywordTok{of}
        \DataTypeTok{Nothing} \OtherTok{{-}\textgreater{}}\NormalTok{ ctorSuffix }\OperatorTok{\textless{}\textgreater{}} \StringTok{".html"}
        \DataTypeTok{Just}\NormalTok{ p }\OtherTok{{-}\textgreater{}}\NormalTok{ ctorSuffix }\OperatorTok{\textless{}/\textgreater{}}\NormalTok{ p}
  \KeywordTok{where}
\OtherTok{    encProd ::}\NormalTok{ (}\DataTypeTok{IsRouteProd}\NormalTok{ xs) }\OtherTok{=\textgreater{}} \DataTypeTok{NP} \DataTypeTok{I}\NormalTok{ xs }\OtherTok{{-}\textgreater{}} \DataTypeTok{K}\NormalTok{ (}\DataTypeTok{Maybe} \DataTypeTok{FilePath}\NormalTok{) xs}
\NormalTok{    encProd }\OtherTok{=}
      \DataTypeTok{K} \OperatorTok{.}\NormalTok{ hcollapseMaybe }\OperatorTok{.}\NormalTok{ hcmap (}\DataTypeTok{Proxy} \OperatorTok{@}\DataTypeTok{IsRoute}\NormalTok{) encTerm}
\OtherTok{    encTerm ::} \DataTypeTok{IsRoute}\NormalTok{ b }\OtherTok{=\textgreater{}} \DataTypeTok{I}\NormalTok{ b }\OtherTok{{-}\textgreater{}} \DataTypeTok{K} \DataTypeTok{FilePath}\NormalTok{ b}
\NormalTok{    encTerm }\OtherTok{=}
      \DataTypeTok{K} \OperatorTok{.}\NormalTok{ encodeRoute }\OperatorTok{.}\NormalTok{ unI}

\OtherTok{datatypeCtors ::} \KeywordTok{forall}\NormalTok{ a}\OperatorTok{.} \DataTypeTok{HasDatatypeInfo}\NormalTok{ a }\OtherTok{=\textgreater{}} \DataTypeTok{NP} \DataTypeTok{ConstructorInfo}\NormalTok{ (}\DataTypeTok{Code}\NormalTok{ a)}
\NormalTok{datatypeCtors }\OtherTok{=}\NormalTok{ constructorInfo }\OperatorTok{$}\NormalTok{ datatypeInfo (}\DataTypeTok{Proxy} \OperatorTok{@}\NormalTok{a)}

\OtherTok{ctorStripPrefix ::} \KeywordTok{forall}\NormalTok{ a}\OperatorTok{.} \DataTypeTok{HasDatatypeInfo}\NormalTok{ a }\OtherTok{=\textgreater{}} \DataTypeTok{ConstructorName} \OtherTok{{-}\textgreater{}} \DataTypeTok{String}
\NormalTok{ctorStripPrefix ctorName }\OtherTok{=}
  \KeywordTok{let}\NormalTok{ name }\OtherTok{=}\NormalTok{ datatypeName }\OperatorTok{$}\NormalTok{ datatypeInfo (}\DataTypeTok{Proxy} \OperatorTok{@}\NormalTok{a)}
   \KeywordTok{in} \FunctionTok{maybe}\NormalTok{ (}\FunctionTok{error} \StringTok{"ctor: bad naming"}\NormalTok{) (T.unpack }\OperatorTok{.}\NormalTok{ T.toLower) }\OperatorTok{$}
\NormalTok{        T.stripPrefix (T.pack }\OperatorTok{$}\NormalTok{ name }\OperatorTok{\textless{}\textgreater{}} \StringTok{"\_"}\NormalTok{) (T.pack ctorName)}
\end{Highlighting}
\end{Shaded}

\texttt{hcollapse} should be familiar; and \texttt{hcmap} is just an
alias of \texttt{hcliftA} (analogous to \texttt{hcliftA2} used in the
above example). New here is \texttt{hcollapseMaybe} which is a custom
version of \texttt{hcollapse} we defined to constrain the number of
products to be either zero or one (as it would not make sense for a
route tye otherwise); its full implementation\footnote{In particular, we
  create a \texttt{HCollapseMaybe} constraint that limits
  \texttt{hcollapse} to work on at most 1 product:

\begin{Shaded}
\begin{Highlighting}[]
\KeywordTok{class} \DataTypeTok{HCollapseMaybe}\NormalTok{ h xs }\KeywordTok{where}
\OtherTok{  hcollapseMaybe ::} \DataTypeTok{SListIN}\NormalTok{ h xs }\OtherTok{=\textgreater{}}\NormalTok{ h (}\DataTypeTok{K}\NormalTok{ a) xs }\OtherTok{{-}\textgreater{}} \DataTypeTok{Maybe}\NormalTok{ a}

\KeywordTok{instance} \DataTypeTok{HCollapseMaybe} \DataTypeTok{NP}\NormalTok{ \textquotesingle{}[] }\KeywordTok{where}
\NormalTok{  hcollapseMaybe \_ }\OtherTok{=} \DataTypeTok{Nothing}

\KeywordTok{instance} \DataTypeTok{HCollapseMaybe} \DataTypeTok{NP}\NormalTok{ \textquotesingle{}[p] }\KeywordTok{where}
\NormalTok{  hcollapseMaybe (}\DataTypeTok{K}\NormalTok{ x }\OperatorTok{:*} \DataTypeTok{Nil}\NormalTok{) }\OtherTok{=} \DataTypeTok{Just}\NormalTok{ x}

\KeywordTok{instance}\NormalTok{ (ps }\OperatorTok{\textasciitilde{}} \DataTypeTok{TypeError}\NormalTok{ (}\DataTypeTok{\textquotesingle{}Text} \StringTok{"Expected at most 1 product"}\NormalTok{)) }\OtherTok{=\textgreater{}} \DataTypeTok{HCollapseMaybe} \DataTypeTok{NP}\NormalTok{ (p \textquotesingle{}}\OperatorTok{:}\NormalTok{ p1 \textquotesingle{}}\OperatorTok{:}\NormalTok{ ps) }\KeywordTok{where}
\NormalTok{  hcollapseMaybe \_ }\OtherTok{=} \DataTypeTok{Nothing} \CommentTok{{-}{-} Unreachable}

\KeywordTok{class}\NormalTok{ (}\DataTypeTok{All} \DataTypeTok{IsRoute}\NormalTok{ xs, }\DataTypeTok{HCollapseMaybe} \DataTypeTok{NP}\NormalTok{ xs) }\OtherTok{=\textgreater{}} \DataTypeTok{IsRouteProd}\NormalTok{ xs}

\KeywordTok{instance}\NormalTok{ (}\DataTypeTok{All} \DataTypeTok{IsRoute}\NormalTok{ xs, }\DataTypeTok{HCollapseMaybe} \DataTypeTok{NP}\NormalTok{ xs) }\OtherTok{=\textgreater{}} \DataTypeTok{IsRouteProd}\NormalTok{ xs}
\end{Highlighting}
\end{Shaded}

  Then we change \texttt{encProd} to be:

\begin{Shaded}
\begin{Highlighting}[]
\OtherTok{encProd ::}\NormalTok{ (}\DataTypeTok{IsRouteProd}\NormalTok{ xs) }\OtherTok{=\textgreater{}} \DataTypeTok{NP} \DataTypeTok{I}\NormalTok{ xs }\OtherTok{{-}\textgreater{}} \DataTypeTok{K}\NormalTok{ (}\DataTypeTok{Maybe} \DataTypeTok{FilePath}\NormalTok{) xs}
\NormalTok{encProd }\OtherTok{=}
  \DataTypeTok{K} \OperatorTok{.}\NormalTok{ hcollapseMaybe }\OperatorTok{.}\NormalTok{ hcmap (}\DataTypeTok{Proxy} \OperatorTok{@}\DataTypeTok{IsRoute}\NormalTok{) encTerm}
\end{Highlighting}
\end{Shaded}

  While propagating the \texttt{All\ IsRouteProd\ (Code\ r)} constraint
  all the way up.} is available in the
\href{https://github.com/srid/generics-sop-examples/blob/master/src/RouteEncoding.hs}{source}.

Finally, we make use of
\href{https://ghc.gitlab.haskell.org/ghc/doc/users_guide/exts/default_signatures.html}{\texttt{DefaultSignatures}}
to provide a default implementation in the \texttt{IsRoute} class:

\begin{Shaded}
\begin{Highlighting}[]
\KeywordTok{class} \DataTypeTok{IsRoute}\NormalTok{ r }\KeywordTok{where}
\OtherTok{  encodeRoute ::}\NormalTok{ r }\OtherTok{{-}\textgreater{}} \DataTypeTok{FilePath}
\NormalTok{  default}\OtherTok{ encodeRoute ::}
\NormalTok{    (}\DataTypeTok{Generic}\NormalTok{ r, }\DataTypeTok{All2} \DataTypeTok{IsRoute}\NormalTok{ (}\DataTypeTok{Code}\NormalTok{ r), }\DataTypeTok{HasDatatypeInfo}\NormalTok{ r) }\OtherTok{=\textgreater{}}
\NormalTok{    r }\OtherTok{{-}\textgreater{}}
    \DataTypeTok{FilePath}
\NormalTok{  encodeRoute }\OtherTok{=}\NormalTok{ gEncodeRoute}
\end{Highlighting}
\end{Shaded}

This, in turn, allows us to derive \texttt{IsRoute} arbitrarily via
\href{https://ghc.gitlab.haskell.org/ghc/doc/users_guide/exts/deriving_strategies.html}{\texttt{DeriveAnyClass}},
which is to say that we get our \texttt{IsRoute} instances for ``free'':

\begin{Shaded}
\begin{Highlighting}[]
\KeywordTok{data} \DataTypeTok{Route}
  \OtherTok{=} \DataTypeTok{Route\_Foo}
  \OperatorTok{|} \DataTypeTok{Route\_Blog} \DataTypeTok{BlogRoute}
  \KeywordTok{deriving}\NormalTok{ stock (}\DataTypeTok{GHC.Generic}\NormalTok{, }\DataTypeTok{Eq}\NormalTok{, }\DataTypeTok{Show}\NormalTok{)}
  \KeywordTok{deriving}\NormalTok{ anyclass (}\DataTypeTok{Generic}\NormalTok{, }\DataTypeTok{HasDatatypeInfo}\NormalTok{, }\DataTypeTok{IsRoute}\NormalTok{)}\OtherTok{\textasciigrave{}}
\end{Highlighting}
\end{Shaded}

\texttt{encodeRoute\ Route\_Foo} now returns \texttt{"foo.html"} and
\texttt{encodeRoute\ \$\ Route\_Blog\ BlogRoute\_Index} returns
\texttt{"blog/index.html"}, all without needing boilerplate
implementation.

\hypertarget{example-3-route-decoding}{%
\subsection{Example 3: route decoding}\label{example-3-route-decoding}}

As a final example, we shall demonstrate what it takes to
\emph{construct} new values. Naturally, our \texttt{IsRoute} class above
needs a new method, \texttt{decodeRoute} for the reverse conversion
(perhaps you want to check the validity of links in the generated HTML):

\begin{Shaded}
\begin{Highlighting}[]
\KeywordTok{class} \DataTypeTok{IsRoute}\NormalTok{ r }\KeywordTok{where}
  \CommentTok{{-}{-} | Encode a route to file path on disk.}
\OtherTok{  encodeRoute ::}\NormalTok{ r }\OtherTok{{-}\textgreater{}} \DataTypeTok{FilePath}
  \CommentTok{{-}{-} | Decode a route from its encoded filepath}
\OtherTok{  decodeRoute ::} \DataTypeTok{FilePath} \OtherTok{{-}\textgreater{}} \DataTypeTok{Maybe}\NormalTok{ r}

\OtherTok{gDecodeRoute ::} \KeywordTok{forall}\NormalTok{ r}\OperatorTok{.}\NormalTok{ (}\DataTypeTok{Generic}\NormalTok{ r, }\DataTypeTok{All2} \DataTypeTok{IsRoute}\NormalTok{ (}\DataTypeTok{Code}\NormalTok{ r), }\DataTypeTok{HasDatatypeInfo}\NormalTok{ r) }\OtherTok{=\textgreater{}} \DataTypeTok{FilePath} \OtherTok{{-}\textgreater{}} \DataTypeTok{Maybe}\NormalTok{ r}
\NormalTok{gDecodeRoute fp }\OtherTok{=} \FunctionTok{undefined}
\end{Highlighting}
\end{Shaded}

\hypertarget{slist}{%
\subsubsection{\texorpdfstring{\texttt{SList}}{SList}}\label{slist}}

Generically constructing values is a little more involved, where it
useful to know about singleton for type-level lists, \texttt{SList}.

\begin{Shaded}
\begin{Highlighting}[]
\KeywordTok{data} \DataTypeTok{SList}\OtherTok{ ::}\NormalTok{ [k] }\OtherTok{{-}\textgreater{}} \DataTypeTok{Type} \KeywordTok{where}
  \DataTypeTok{SNil}\OtherTok{  ::} \DataTypeTok{SList}\NormalTok{ \textquotesingle{}[]}
  \DataTypeTok{SCons}\OtherTok{ ::} \DataTypeTok{SListI}\NormalTok{ xs }\OtherTok{=\textgreater{}} \DataTypeTok{SList}\NormalTok{ (x \textquotesingle{}}\OperatorTok{:}\NormalTok{ xs)}

\CommentTok{{-}{-} | Get hold of an explicit singleton (that one can then}
\CommentTok{{-}{-} pattern match on) for a type{-}level list}
\CommentTok{{-}{-}}
\OtherTok{sList ::} \DataTypeTok{SListI}\NormalTok{ xs }\OtherTok{=\textgreater{}} \DataTypeTok{SList}\NormalTok{ xs}
\NormalTok{sList }\OtherTok{=} \OperatorTok{...}
\end{Highlighting}
\end{Shaded}

We require heavy use of \texttt{sList} to generically implement
\texttt{decodeRoute}. \texttt{sList} pretty much allows us to
``case-match'' on the type-level list and build our combinators
accordingly, as we will see below.

\hypertarget{anamomrphism-combinators}{%
\subsubsection{Anamomrphism
combinators}\label{anamomrphism-combinators}}

To implement \texttt{decodeRoute} generically, we are looking to
construct a \texttt{NS\ (NP\ I)\ (Code\ r)} depending on which
constructor the first path segment of \texttt{fp} matches with. Then, we
recurse into constructing the inner route for the sum constructor's
(only and optional) product type. This recursive building of values is
called
\href{https://en.wikipedia.org/wiki/Anamorphism\#Anamorphisms_in_functional_programming}{anamorphism}.
In particular, we need two anamorphisms: one for the outer sum and
another for the inner product.

\texttt{generics-sop} already provides
\href{https://hackage.haskell.org/package/sop-core-0.5.0.2/docs/Data-SOP-NS.html\#v:cana_NS}{\texttt{cana\_NS}}
and
\href{https://hackage.haskell.org/package/sop-core-0.5.0.2/docs/Data-SOP-NP.html\#v:cana_NP}{\texttt{cana\_NP}}
as anamorphisms for \texttt{NS} and \texttt{NP} respectively. However we
need a slightly different version of them, to return \texttt{Maybe}
values instead. We shall define them (prefixed with \texttt{m})
accordingly as follows (note the use of \texttt{sList}):

\begin{Shaded}
\begin{Highlighting}[]
\CommentTok{{-}{-} | Like \textasciigrave{}cana\_NS\textasciigrave{} but returns a Maybe}
\OtherTok{mcana\_NS ::}
  \KeywordTok{forall}\NormalTok{ c proxy s f xs}\OperatorTok{.}
\NormalTok{  (}\DataTypeTok{All}\NormalTok{ c xs) }\OtherTok{=\textgreater{}}
\NormalTok{  proxy c }\OtherTok{{-}\textgreater{}}
\NormalTok{  (}\KeywordTok{forall}\NormalTok{ y ys}\OperatorTok{.}\NormalTok{ c y }\OtherTok{=\textgreater{}}\NormalTok{ s (y \textquotesingle{}}\OperatorTok{:}\NormalTok{ ys) }\OtherTok{{-}\textgreater{}} \DataTypeTok{Either}\NormalTok{ (}\DataTypeTok{Maybe}\NormalTok{ (f y)) (s ys)) }\OtherTok{{-}\textgreater{}}
\NormalTok{  s xs }\OtherTok{{-}\textgreater{}}
  \DataTypeTok{Maybe}\NormalTok{ (}\DataTypeTok{NS}\NormalTok{ f xs)}
\NormalTok{mcana\_NS \_ decide }\OtherTok{=}\NormalTok{ go sList}
  \KeywordTok{where}
\OtherTok{    go ::} \KeywordTok{forall}\NormalTok{ ys}\OperatorTok{.}\NormalTok{ (}\DataTypeTok{All}\NormalTok{ c ys) }\OtherTok{=\textgreater{}} \DataTypeTok{SList}\NormalTok{ ys }\OtherTok{{-}\textgreater{}}\NormalTok{ s ys }\OtherTok{{-}\textgreater{}} \DataTypeTok{Maybe}\NormalTok{ (}\DataTypeTok{NS}\NormalTok{ f ys)}
\NormalTok{    go }\DataTypeTok{SNil}\NormalTok{ \_ }\OtherTok{=} \DataTypeTok{Nothing}
\NormalTok{    go }\DataTypeTok{SCons}\NormalTok{ s }\OtherTok{=} \KeywordTok{case}\NormalTok{ decide s }\KeywordTok{of}
      \DataTypeTok{Left}\NormalTok{ x }\OtherTok{{-}\textgreater{}} \DataTypeTok{Z} \OperatorTok{\textless{}$\textgreater{}}\NormalTok{ x}
      \DataTypeTok{Right}\NormalTok{ s\textquotesingle{} }\OtherTok{{-}\textgreater{}} \DataTypeTok{S} \OperatorTok{\textless{}$\textgreater{}}\NormalTok{ go sList s\textquotesingle{}}

\CommentTok{{-}{-} | Like \textasciigrave{}cana\_NP\textasciigrave{} but returns a Maybe}
\OtherTok{mcana\_NP ::}
  \KeywordTok{forall}\NormalTok{ c proxy s f xs}\OperatorTok{.}
\NormalTok{  (}\DataTypeTok{All}\NormalTok{ c xs) }\OtherTok{=\textgreater{}}
\NormalTok{  proxy c }\OtherTok{{-}\textgreater{}}
\NormalTok{  (}\KeywordTok{forall}\NormalTok{ y ys}\OperatorTok{.}\NormalTok{ (c y, }\DataTypeTok{SListI}\NormalTok{ ys) }\OtherTok{=\textgreater{}}\NormalTok{ s (y \textquotesingle{}}\OperatorTok{:}\NormalTok{ ys) }\OtherTok{{-}\textgreater{}} \DataTypeTok{Maybe}\NormalTok{ (f y, s ys)) }\OtherTok{{-}\textgreater{}}
\NormalTok{  s xs }\OtherTok{{-}\textgreater{}}
  \DataTypeTok{Maybe}\NormalTok{ (}\DataTypeTok{NP}\NormalTok{ f xs)}
\NormalTok{mcana\_NP \_ uncons }\OtherTok{=}\NormalTok{ go sList}
  \KeywordTok{where}
\OtherTok{    go ::} \KeywordTok{forall}\NormalTok{ ys}\OperatorTok{.}\NormalTok{ (}\DataTypeTok{All}\NormalTok{ c ys) }\OtherTok{=\textgreater{}} \DataTypeTok{SList}\NormalTok{ ys }\OtherTok{{-}\textgreater{}}\NormalTok{ s ys }\OtherTok{{-}\textgreater{}} \DataTypeTok{Maybe}\NormalTok{ (}\DataTypeTok{NP}\NormalTok{ f ys)}
\NormalTok{    go }\DataTypeTok{SNil}\NormalTok{ \_ }\OtherTok{=} \FunctionTok{pure} \DataTypeTok{Nil}
\NormalTok{    go }\DataTypeTok{SCons}\NormalTok{ s }\OtherTok{=} \KeywordTok{do}
\NormalTok{      (x, s\textquotesingle{}) }\OtherTok{\textless{}{-}}\NormalTok{ uncons s}
\NormalTok{      xs }\OtherTok{\textless{}{-}}\NormalTok{ go sList s\textquotesingle{}}
      \FunctionTok{pure} \OperatorTok{$}\NormalTok{ x }\OperatorTok{:*}\NormalTok{ xs}
\end{Highlighting}
\end{Shaded}

\hypertarget{implement-gdecoderoute}{%
\subsubsection{\texorpdfstring{Implement
\texttt{gDecodeRoute}}{Implement gDecodeRoute}}\label{implement-gdecoderoute}}

Now we are ready to use a combination of \texttt{sList},
\texttt{mcana\_NS} and \texttt{mcana\_NP} to implement
\texttt{gDecodeRoute}:

\begin{Shaded}
\begin{Highlighting}[]
\OtherTok{gDecodeRoute ::} \KeywordTok{forall}\NormalTok{ r}\OperatorTok{.}\NormalTok{ (}\DataTypeTok{Generic}\NormalTok{ r, }\DataTypeTok{All} \DataTypeTok{IsRouteProd}\NormalTok{ (}\DataTypeTok{Code}\NormalTok{ r), }\DataTypeTok{All2} \DataTypeTok{IsRoute}\NormalTok{ (}\DataTypeTok{Code}\NormalTok{ r), }\DataTypeTok{HasDatatypeInfo}\NormalTok{ r) }\OtherTok{=\textgreater{}} \DataTypeTok{FilePath} \OtherTok{{-}\textgreater{}} \DataTypeTok{Maybe}\NormalTok{ r}
\NormalTok{gDecodeRoute fp }\OtherTok{=} \KeywordTok{do}
  \CommentTok{{-}{-} We operate on first element of the filepath, and inductively decode the rest.}
\NormalTok{  basePath }\OperatorTok{:}\NormalTok{ restPath }\OtherTok{\textless{}{-}} \FunctionTok{pure} \OperatorTok{$}\NormalTok{ splitDirectories fp}
  \CommentTok{{-}{-} Build the sum using an anamorphism}
\NormalTok{  to }\OperatorTok{.} \DataTypeTok{SOP}
    \OperatorTok{\textless{}$\textgreater{}}\NormalTok{ mcana\_NS }\OperatorTok{@}\DataTypeTok{IsRouteProd} \OperatorTok{@}\NormalTok{\_ }\OperatorTok{@}\NormalTok{\_ }\OperatorTok{@}\NormalTok{(}\DataTypeTok{NP} \DataTypeTok{I}\NormalTok{)}
      \DataTypeTok{Proxy}
\NormalTok{      (anamorphismSum basePath restPath)}
\NormalTok{      (datatypeCtors }\OperatorTok{@}\NormalTok{r)}
  \KeywordTok{where}
    \CommentTok{{-}{-} The \textasciigrave{}base\textasciigrave{} part of the path should correspond to the constructor name. }
\OtherTok{    anamorphismSum ::} \KeywordTok{forall}\NormalTok{ xs xss}\OperatorTok{.} \DataTypeTok{IsRouteProd}\NormalTok{ xs }\OtherTok{=\textgreater{}} \DataTypeTok{FilePath} \OtherTok{{-}\textgreater{}}\NormalTok{ [}\DataTypeTok{FilePath}\NormalTok{] }\OtherTok{{-}\textgreater{}} \DataTypeTok{NP} \DataTypeTok{ConstructorInfo}\NormalTok{ (xs \textquotesingle{}}\OperatorTok{:}\NormalTok{ xss) }\OtherTok{{-}\textgreater{}} \DataTypeTok{Either}\NormalTok{ (}\DataTypeTok{Maybe}\NormalTok{ (}\DataTypeTok{NP} \DataTypeTok{I}\NormalTok{ xs)) (}\DataTypeTok{NP} \DataTypeTok{ConstructorInfo}\NormalTok{ xss)}
\NormalTok{    anamorphismSum base rest (p }\OperatorTok{:*}\NormalTok{ ps) }\OtherTok{=}
\NormalTok{      fromMaybe (}\DataTypeTok{Right}\NormalTok{ ps) }\OperatorTok{$} \KeywordTok{do}
        \KeywordTok{let}\NormalTok{ ctorSuffix }\OtherTok{=}\NormalTok{ ctorStripPrefix }\OperatorTok{@}\NormalTok{r (constructorName p)}
        \DataTypeTok{Left} \OperatorTok{\textless{}$\textgreater{}} \KeywordTok{case}\NormalTok{ sList }\OperatorTok{@}\NormalTok{xs }\KeywordTok{of}
          \DataTypeTok{SNil} \OtherTok{{-}\textgreater{}} \KeywordTok{do}
            \CommentTok{{-}{-} For constructors without arguments, we simply expect the \textasciigrave{}rest\textasciigrave{}}
            \CommentTok{{-}{-} of the path to be empty.}
\NormalTok{            guard }\OperatorTok{$}\NormalTok{ ctorSuffix }\OperatorTok{\textless{}\textgreater{}} \StringTok{".html"} \OperatorTok{==}\NormalTok{ base }\OperatorTok{\&\&} \FunctionTok{null}\NormalTok{ rest}
            \FunctionTok{pure} \OperatorTok{$} \DataTypeTok{Just} \DataTypeTok{Nil}
          \DataTypeTok{SCons} \OtherTok{{-}\textgreater{}} \KeywordTok{do}
            \CommentTok{{-}{-} For constructors with an argument, we ensure that the constructor}
            \CommentTok{{-}{-} name matches the base part and then recurse into decoding the}
            \CommentTok{{-}{-} argument itself.}
\NormalTok{            guard }\OperatorTok{$}\NormalTok{ ctorSuffix }\OperatorTok{==}\NormalTok{ base}
            \FunctionTok{pure} \OperatorTok{$}
\NormalTok{              mcana\_NP }\OperatorTok{@}\NormalTok{\_ }\OperatorTok{@}\NormalTok{\_ }\OperatorTok{@}\NormalTok{\_ }\OperatorTok{@}\DataTypeTok{I}
\NormalTok{                (}\DataTypeTok{Proxy} \OperatorTok{@}\DataTypeTok{IsRoute}\NormalTok{)}
\NormalTok{                anamorphismProduct}
                \DataTypeTok{Proxy}
      \KeywordTok{where}
\OtherTok{        anamorphismProduct ::} \KeywordTok{forall}\NormalTok{ y1 ys1}\OperatorTok{.}\NormalTok{ (}\DataTypeTok{IsRoute}\NormalTok{ y1, }\DataTypeTok{SListI}\NormalTok{ ys1) }\OtherTok{=\textgreater{}} \DataTypeTok{Proxy}\NormalTok{ (y1 \textquotesingle{}}\OperatorTok{:}\NormalTok{ ys1) }\OtherTok{{-}\textgreater{}} \DataTypeTok{Maybe}\NormalTok{ (}\DataTypeTok{I}\NormalTok{ y1, }\DataTypeTok{Proxy}\NormalTok{ ys1)}
\NormalTok{        anamorphismProduct }\DataTypeTok{Proxy} \OtherTok{=} \KeywordTok{case}\NormalTok{ sList }\OperatorTok{@}\NormalTok{ys1 }\KeywordTok{of}
          \CommentTok{{-}{-} We "case match" on the rest of the products, to handle the scenario}
          \CommentTok{{-}{-} of there being exactly one product.}
          \DataTypeTok{SNil} \OtherTok{{-}\textgreater{}} \KeywordTok{do}
            \CommentTok{{-}{-} Recurse into the only product argument}
\NormalTok{            guard }\OperatorTok{$} \FunctionTok{not} \OperatorTok{$} \FunctionTok{null}\NormalTok{ rest}
\NormalTok{            r\textquotesingle{} }\OtherTok{\textless{}{-}}\NormalTok{ decodeRoute }\OperatorTok{@}\NormalTok{y1 }\OperatorTok{$}\NormalTok{ joinPath rest}
            \FunctionTok{pure}\NormalTok{ (}\DataTypeTok{I}\NormalTok{ r\textquotesingle{}, }\DataTypeTok{Proxy}\NormalTok{)}
          \DataTypeTok{SCons} \OtherTok{{-}\textgreater{}}
            \CommentTok{{-}{-} Not reachable, due to HCollapseMaybe constraint}
            \DataTypeTok{Nothing}
\end{Highlighting}
\end{Shaded}

We split the path \texttt{fp} and process the first path segment by
matching it with one of the sum constructors. In \texttt{anamorphismSum}
we handle the two cases of null product constructor and singleton
product constructor (\texttt{mcana\_NS} is responsible for recursing
into other sum constructors). For null product, we match the file path
with ``\$\{constructorSuffix\}.html'' and return immediately. For a
single product case, we use \texttt{mcana\_NP} to build the product.
\texttt{anammorphismProduct} uses \texttt{sList} to case match on the
rest of the products (i.e.~2nd, etc.) - and calls \texttt{decodeRoute}
on the first product only if the rest is empty, which in turn requires
us to the \texttt{IsRoute} constraint all the way above.

Finally, we use \texttt{DefaultSignatures} to specify a default
implementation in \texttt{IsRoute} class.

\hypertarget{putting-it-all-together}{%
\subsubsection{Putting it all together}\label{putting-it-all-together}}

We can test that our code works in ghci:

\begin{Shaded}
\begin{Highlighting}[]
\OperatorTok{\textgreater{}} \KeywordTok{import} \DataTypeTok{RouteEncoding} 
\OperatorTok{\textgreater{}}\NormalTok{ encodeRoute }\DataTypeTok{Route\_Index}
\StringTok{"index.html"}
\OperatorTok{\textgreater{}}\NormalTok{ decodeRoute }\OperatorTok{@}\DataTypeTok{Route} \OperatorTok{$}\NormalTok{ encodeRoute }\DataTypeTok{Route\_Index}
\DataTypeTok{Just} \DataTypeTok{Route\_Index}
\end{Highlighting}
\end{Shaded}

To be completely sure, we can test it with inductive route values:

\begin{Shaded}
\begin{Highlighting}[]
\OperatorTok{\textgreater{}}\NormalTok{ encodeRoute }\OperatorTok{$} \DataTypeTok{Route\_Blog} \OperatorTok{$} \DataTypeTok{BlogRoute\_Post} \StringTok{"hello"}
\StringTok{"blog/post/hello.html"}
\OperatorTok{\textgreater{}}\NormalTok{ decodeRoute }\OperatorTok{@}\DataTypeTok{Route} \StringTok{"blog/post/hello.html"}
\DataTypeTok{Just}\NormalTok{ (}\DataTypeTok{Route\_Blog}\NormalTok{ (}\DataTypeTok{BlogRoute\_Post} \StringTok{"hello"}\NormalTok{))}
\OperatorTok{\textgreater{}} 
\end{Highlighting}
\end{Shaded}

This concludes our introduction to \texttt{generics-sop}.

\hypertarget{further-information}{%
\subsection{Further information}\label{further-information}}

\begin{itemize}
\tightlist
\item
  \href{https://github.com/srid/generics-sop-examples/}{Source code} for
  this blog post
\item
  \href{https://www.youtube.com/watch?v=sQxH349HOik}{This ZuriHac talk}
  provides a good introduction to generics-sop
\item
  \href{https://github.com/kosmikus/SSGEP/blob/master/LectureNotes.pdf}{Applying
  Type-Level and Generic Programming in Haskell} by Andres Löh acts as a
  lengthy tutorial cum documentation for generics-sop
\end{itemize}
